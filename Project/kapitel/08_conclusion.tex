\section{Conclusion}

In this project, we aimed to find correlations between Lovecraft's personal life events and the main 
themes of his writings. For this purpose, we used the digital humanities, particularly topic modeling, 
to gather the main topics across Lovecraft's works and their respective share for each work. We also 
used a web interface to find relevant terms for each topic, this added further depth to the insights 
we got.\\

The research question could be answered with good satisfaction since we clearly found patterns in 
the represented model that correlate to events of his life. Such include his mother's death, the 
move to Brooklyn, his return to Providence, and divorce, as well as his aunt's death. Though we 
expected some even more visible patterns, we found a good amount of links to these events when 
comparing the dates his works were written with the dates of the events. It's also necessary to 
note that for some topics, the correlating event like his mother's death or his divorce may have 
impacted some works even earlier. This is because the writing of a story is not completed in a 
short time and we took this into account when discussing his divorce's influence, which may have 
been a topic for him earlier to the official divorce with problems in the relationship. We also 
got some key insights into one of the most important motifs of Lovecraft. Contrary to popularity, 
the most important works do not cover the 'Cthulhu Mythos', which Lovecraft is most known for today, 
but the 'Dream Cycle' and Lovecraft's personal input like his supposed alter ego Randolph Carter, 
his love for cats, and the dreams that influenced his writings.\\

From the distribution of topics and their relevant terms, it's also slightly observable when 
Lovecraft moved to Brooklyn and returned to Providence, marking a break in his dream cycle and 
a return to more cosmic horror and self-fulfillment when he was back in his hometown. The last 
few works cover a mix of topics with no clear patterns that could also be described by his shift 
towards ghostwriting and fictional stories becoming less frequent and richer in length. Social 
events of that time, like World War 1 or the great depression could not be clearly linked to and 
were therefore not covered.\\

It's important to mention that this project is just the foundation of possible research on his 
life-work correlation using digital methods like LDA and machine learning. Future work regarding 
the digital humanities may not only include topic modeling on his fictional works but also his 
non-fiction and letters. It's also interesting to bring up stylometry as a method here, where 
we could see shifts in his stylistic devices that may correlate to personal events of his as 
well. Overall, this project has shown the potential of topic modeling as a tool for literary 
analysis and its ability to uncover previously unknown patterns in a body of text. It contributes 
to the ongoing research of Lovecraft's works in the digital age and offers notable insights and 
foundations for such future work.
