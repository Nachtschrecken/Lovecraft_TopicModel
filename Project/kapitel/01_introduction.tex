\section{Introduction}

This project aims to analyze the works of Howard Phillips Lovecraft, one of the most known horror 
authors, regarding their themes and patterns. This will be done using digital humanities and topic 
modeling on all of his writings to collect a timeline-like collection of the most common themes in 
his works. H.P. Lovecraft highly contributed to the modern horror genre with his detailed descriptions 
of cosmic horror (a term defined by his writings) and stories about monsters and scary events. While 
his writings are known for complexity and style, his thematic and contextual depth is equally impressive 
and the subject of this project. Therefore, Lovecraft is well-known by horror and science fiction fans, 
like Stephen King, who stated that Lovecraft heavily influenced his style and ideas.\\

Even though Lovecraft lived from 1890-1937 and did not encounter computers or other devices of our 
current age, his works and letters are digitalized and preserved for reading and research. One key 
aspect of this project is to apply the digital humanities method of topic modeling to all of 
Lovecraft’s works. The digital humanities is a field of study, research, teaching, and invention 
concerned with the intersection of computing and the disciplines of the humanities. One can think 
of it as a bridge between the two cultures of natural science, which tries to explain what is going 
on, and humanities, which tries to understand what is happening. One key figure in this field is 
Roberto Busa (1913 - 2011), who used computational methods to create a collection of all the words 
of Thomas Aquinas, later known as the Index Thomisticus. He achieved his goal with the help of IBM, 
at that time one of the biggest computer manufacturers, and connected his studies with the help of 
digital computing to process the millions of words he strived to process.\\

By using the digital humanities, we can gain new insights into Lovecraft’s writings and identify 
recurring themes, patterns, and motifs in the big corpus of it. The methods used in the digital 
humanities include Stylometry, Topic Modeling, Network Analysis, and Geovisualization. For the 
purpose of this project, we will use topic modeling to gather insight into the large corpus and 
structure the results like a timeline for every one of his works. Regarding digital humanities, 
this project aims to explore the humanities of literature and apply computational methods to it. 
That way we can see the most occurring themes from his first works to his last ones and compare 
them in terms of lore consistency (how did his themes and mythos change?), personal correlation 
(does he refer to personal life events of his time?) and comprehensive insight on the different 
features of his works. Some works already covered features of his writings and life like the fear 
of progress, the dangers of scientific progress, and the persistence of ancient evils.
