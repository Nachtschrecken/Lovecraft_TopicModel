\section{Data Overview}

The data for this project can be found on Kaggle. We decided to go with this dataset because 
it features all of Lovecraft’s written texts in .txt. format, which makes it easy to process 
the data. All stories exist in public domain and are not subject to copyright. Due to the 
stories being public domain, and more datasets being available online as well as raw HTML 
texts available, we had many sources to choose from. A comparison with other sources and 
physical prints of stories does not indicate any biases or alteration in the texts. Though 
we have found some works in the dataset that are not solely written by Lovecraft, since he 
collaborated with many other authors of his time and even took ghostwriting commissions. 
Another problem also occured with other datasets, where the earliest works of Lovecraft 
are missing. We decided to still go with this dataset and remove works which we don’t want 
to include. Since our research question wants to only cover Lovecraft’s ideas and writings, 
the influence from other authors or commissioners could alter the results. That’s because 
the ideas of Lovecraft are not really present in these specific writings, though his 
stylometry may be the same.\\

The dataset contains 102 works of Lovecraft, including all collaborations. As we just 
mentioned, collaborations, revisions and ghost writings are not subject of this project 
and will be removed. An overview of all the works written solely by Lovecraft, his fiction, 
can be found in his bibliography on Wikipedia. In this case only his ‘fiction’ will be 
included in this project as well as some of his works found under ‘Juvenilia’, which 
include his earliest writings when he was young. Also not included in this project are 
his poetry works, philosophical works and scientific works since they are either too 
short or provide no information in regard to his mythos and lore. After only keeping 
the works of Lovecraft from the bibliography the dataset has remaining 70 stories. 
These works can be distinguished into short stories, novellas, or even fragments from 
letters or novellas. Lovecraft’s works largely vary in length, some just a few pages, 
and other whole novellas. Fragments could also contain just a few paragraphs. One example 
is the story ‘Azatoth’ (written June 1922), which is very short in length and was 
supposed to be part of a whole novella, though only this fragment survived. It is currently 
unknown how many such fragments were lost, but his wife admitted to have burned a lot of 
his letters. If these included other fragments or whole works of Lovecraft not yet 
published or copied at that time, that means there were works which are lost forever.\\

The 70 works in this modified dataset are .txt documents. The text files contain the 
title and the raw text. We formatted the files using R and pasted them into a .csv 
format. In this process we also added the dates as a second row to the text and manually 
put another line between paragraphs process them using regular expressions. The file to 
process the dataset can be found in the repository as ‘format.r’. The .csv file contains 
for every text the text\_id, then for every paragraph of the text the doc\_id, the title, 
the date, and the paragraph containing the text. This way we can later use topic modeling 
on the csv to create a timeline-like plot of the main themes. The dates found in different 
sources can be misleading, since some stories were published after Lovecraft’s death. 
One example is the story ‘The Case of Charles Dexter Ward’, which was written in 1927, 
partly published in 1941, and fully published in 1943, while Lovecraft already died in 
1937. The dates we manually added to each file represents the date a work was written 
as provided on WikiSource. This corresponds better to his personal life and events. 
For dates where only the year is provided (see ‘The Alchemist’, 1908) or no explicit 
month (see ‘The Street’, late 1919) we used proper months according to quarters or the 
beginning of seasons. For plain years we added the January of that year. Taking into 
account publishing dates is no option, since they do not reflect Lovecraft’s personal 
events, may not be in order, or even have been published after his death as already 
mentioned.