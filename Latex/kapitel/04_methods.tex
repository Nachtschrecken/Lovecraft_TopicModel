\section{Method Overview}

In the digital humanities, there are four main methods to work on data and achieve information. 
Stylometry is a method to find textual similarities between texts, Network analysis can be used 
to find relations in data, Geovisualization is used to representing data on maps, and the fourth 
method, which we will use for this project, is topic modeling. Topic modeling is a method to 
gather information from texts regarding the content. Using this computational approach we can 
uncover the main themes and topics from texts, not just by using a frequentist method with the 
most words used, but by putting context in the found information. A topic model achieves this 
while it discovers the degree to which each document exhibits those topics. That way we can 
build a statistical lens that encodes our specific knowledge, theories, and assumptions about 
texts \cite{blei}.\\

Using topic modeling on the collection of Lovecraft's works will help us uncover the most 
important themes per text regarding not only the frequentist quantity of certain topics but, 
as already mentioned, the contextual frequency. We want to know what topic dominates each text, 
and topic modeling is exactly the tool we need for that. By writing short scripts in the 
programming language R, we can compute the main topics easily and even visualize them in plots. 
For example, the R package \textit{ggplot2} is an easy tool to plot the results in a timeline-style plot. \\

We will apply topic modeling to each text. This is done because if we would use chunks of text 
over the whole concatenated collection, we would lose the important context. Especially in this 
literature of fantasy and horror, we need to keep the topics regarding their text. In one text 
like 'The Thing on the Doorstep' (written 1933) the main topic may refer to \textit{creature,
darkness, house} or others. But another text like 'The Nameless City' (written 1921) may 
yield topics like \textit{place, sculpture} and \textit{ancient}. When using chunks of a concatenated 
collection of his writings, we would use the context of each text. Therefore we work on each 
text separately, even if some texts are significantly shorter than others.\\

To reflect on some biases with this approach, we have to particularly emphasize this problem: 
Some works of Lovecraft are shorter than others. Though we do not expect any major differences 
in contextual topics between short and long texts, this has to be noted. A measure we use to 
counter this is computing a topic model using paragraphs of similar size for every work. 
According to the word count of Lovecraft's works the length of his works linearly increases 
with the years of writing. Another major bias in topic modeling is the approach of Natural 
Language Processing (NLP). For our purpose, we will work with English libraries, stopwords, 
and dictionaries. This could lead to problems when it comes to topics that do not reflect the 
English language. Lovecraft is known for his monsters and cosmic words like \textit{Necronomicon} or 
\textit{Cthulhu}. For this, we will focus on topics and words that are subject to the English language 
and can be covered by NLP. We will also keep in mind that in Lovecraft's style, specific words 
tend to occur more often than others in the literary context.