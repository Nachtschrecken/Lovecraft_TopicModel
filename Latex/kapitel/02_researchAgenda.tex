\section{Research Agenda}

The research agenda of this project is 'Is there a correlation between H.P. Lovecraft's personal 
life events and the themes in his works of that time?'. The goal is to understand how they relate 
to Lovecraft's personal life experiences, eventually also historical events, cultural and social 
influences, as well as to the broader literary and intellectual traditions of his time. To answer 
that question we will apply topic modeling to each of his works in chronological order. By 
comparing the resulting main topics we aim to find changes in the lore of his mythos as well 
as correlations to the main events in his life.\\

We will use topic modeling on a dataset of Lovecraft's works to represent the main themes of his 
writings from a corpus of all of his works. Because Lovecraft died over 70 years ago, many of his 
writings are public domain and can easily be accessed from various institutions providing good 
coverage for the material. Nevertheless, some of the works he contributed are still not public 
domain, since some authors he worked with lived on longer than he did (e.g. Clarke Ashton Smith 
died 1961). This is part of the reason why we decided only to cover works solely written by 
Lovecraft himself. The more important reason is that mixing up his writings with ghostwritten or 
co-written works could alter the results since we only want to cover Lovecraft's themes. Using the 
programming language R, we first format the corpus into a collection of all works, then process 
the texts to remove annotations and unnecessary titles and use topic modeling on the corpus to 
gather and represent the data. After that, we will analyze the gathered data represented as a 
plotted diagram in R. That way we can create a timeline-like graph of the main themes and motifs 
in Lovecraft's works and discuss them in retrospect to events of that time or his personal life.\\

To properly find correlations between his writings and private life or social events of his time, 
we will shortly examine personal events from his letters as well as his biographical records. After 
that, we include the ideas and insight into the discussion. We expect to find results regarding 
tragic events like his mother's death or homesickness to Providence during his stay in Brooklyn, 
which heavily influenced his writings since Lovecraft is known for his rejection of modern 
movements of that time. He was known for his dislike of New York City, after visiting the city 
during his stay in Brooklyn. This is one of the interesting parts of his life we expect to find 
reflected in his writings. We already know of the story 'The Horror at Red Hook' (written 1925) 
in which Lovecraft negatively reflects on Red Hook, Brooklyn. It is also important to mention 
that Lovecraft condemned ethnic groups he encountered during his time in New York, mainly 
Afro-Americans and Asian immigrants. It is no secret that Lovecraft used racist slurs and sometimes 
even antisemitic stereotypes in his writings \cite{romano}, all of which will be analyzed according to his 
personal events to explore how his prejudices shaped his writing. For all of these aspects of his 
life, we aim to find similar patterns in his writing through which one could assign a batch of 
his works to a phase of his life.