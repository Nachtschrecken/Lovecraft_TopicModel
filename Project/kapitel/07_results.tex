\section{Results and Discussion}

For the results of this Project, we represented the retrieved topic model in plots. For one we 
used a simple plot package, ggplot in R, to represent every of Lovecraft’s works in chronological 
order with their distribution of the topics. We also used the R package LDAvis to show correlations 
between the topics and use an interactive interface to see the importance of relevant terms with 
respect to a topic. The resulting topics are as follows:

\begin{enumerate}
    \itemsep0em
    \item thing, time, feel, mind, life
    \item thing, horror, eye, black, sound
    \item dream, city, day, night, strange
    \item room, door, hand, floor, window
    \item place, wall, stone, vast, foot
    \item west, specimen, work, lake, body
    \item letter, curwen, time, late, charles
    \item street, house, hill, place, town
    \item family, child, son, home, time
    \item thing, folk, time, good, obed
    \item voice, sound, begin, word, time
    \item tomb, night, fear, thing, remain
    \item land, city, sea, dream, water
    \item thing, night, place, talk, people
    \item carter, ghoul, cat, night-gaunts, ship
\end{enumerate}

\begin{figure}[h]
    \centering
    \includegraphics[width=1\textwidth]{/Users/ferriskleier/PECK/DigitalHumanities/Lovecraft_TopicModel/Project/images/plot.png}
    \caption{The plot assigning the proportion of a topic to each story of Lovecraft}
    \label{fig:mesh1}
\end{figure}

Figure 1 shows the bar chart we created to plot every of Lovecraft’s stories in chronological order 
with the percentage of the topic they share. On the x-axis, you can see the number of the story. 
Keep in mind that we used the date Lovecraft wrote the story and changed unspecified dates like just 
a year to the January of that year. For the enumeration of his works to compare to this figure, see 
stories\_list.txt. On the y-axis, you can see the distribution in percentage. Each bar corresponds to 
one work and the percentage of a topic across each of the topics in that work. The order for each 
topic per entry is the same, it’s not ordered by percentage but by topic. On the right-hand side, you 
can find the colors for each topic, pink being topic 01 consisting of the words ‘thing, time, feel, 
mind, life’.\\

The bar chart clearly shows that some topics dominate a story and others have a fairly distributed 
amount of topics. For example, the first topic in pink is highly present in some works. For 
work 55, ‘The Dream Quest of Unknown Kadath’ (written 1927) one can see that topic 15 is highly 
present, which satisfies the expectation because it is a fairly long work with a unique setting 
and motif. Another example is work 63, ‘At the Mountains of Madness’ (written 1931), the topics 
5 and 6 are more present than in any other work, which again satisfies the expectation for this 
work being unique in the antarctic setting and length. The bar chart also shows batches of topics 
for several time frames, like his works 30 to 38 which are dominated by topic 3. After reviewing 
the topic distribution for every work, we were very satisfied with the results and decided to use 
this resulting topic model for further examination.\\

To put this into context with Lovecraft’s personal life, we took several events that may have had 
a significant influence on him and checked for the topics of that time. We reviewed ‘An Epicure in 
the Terrible’ (Edited by David E. Schultz and S.T. Joshi) which is a perfect collection of essays 
by different authors regarding Lovecraft’s life. Topic 2 being blue clearly shows the time Lovecraft 
began to write the ‘Dream Cycle’, since the topics correlate well with the stories. For work 10, 
‘Polaris’ (written 1918), and work 14, ‘The White Ship’ (written 1919), one can see the high share 
of topic 3, which just continues for later stories in the dream cycle as well. This is due to the 
fact that around this time, Lovecraft met Lord Dunsany, who Lovecraft admired and who heavily 
influenced his early works which led to the dream cycle. For the works 24 (‘Nyarlathotep’, 
written 1920), 26 (‘From Beyond’, written 1920), and 29 (‘The Nameless City’, written 1921) 
topic 2 is highly present and these works mark the first shift in Lovecraft’s writings towards 
cosmic horror, which later caused the works of the ‘Cthulhu Mythos’. Starting from work 30, ‘The 
Quest of Iranon’ (written 1921), topic 3 dominates parts of his writings. Topic 3 is a good indicator 
for writings covering the dream cycle, which matches the works. The first interesting correlation 
between Lovecraft’s personal life and the graph can be seen starting from work 33, ‘The Outsider’ 
(written 1921), with topic 2 (blue) having a great share for some works and before. This may relate 
to the death of his mother in May 1921. Not only did he not write for a short period, but topic 2 
consists of the keywords ‘thing, horror, eye, black, sound’ which suggests a coping to his mother’s 
death as well as the time before may having an impact on this topic as well because she was admitted 
to a hospital. Lovecraft moved to Brooklyn in 1922 and married his wife, Sonia Haft Greene, which 
can be seen in a clear shift in topics during the time around work 39 and following, putting a break 
on the dream cycle and prompting topic 12 to be more present. Topic 12 ends being more present starting 
from work 44, ‘The Festival’ (written October 1923). That’s interesting because he did also write just 
one story for almost two years during that time, which goes in hand with his marriage and time in New 
York. Lovecraft started writing again when in 1925 he moved to Red Hook, Brooklyn, living alone 
because his wife worked somewhere else. It is known that Lovecraft despised Red Hook and was 
negatively standing against minorities that lived there. The only observable difference is, that 
topics 4, 5, and 9 started to have a stable share starting from that time. This started with his 
first story after the break ‘The Horror at Red Hook’ (written 1925) which strongly suggests his 
cope with living there. The next observable pattern begins with work 52, ‘The Strange High House 
in the Mist’ (written 1926), which is represented with a higher amount of topic 3 being the topic 
that indicates the continuation of the dream cycle. This is very interesting, because Lovecraft 
moved back to Providence in April 1926, resulting from his growing homesickness and that he became 
increasingly depressed by his isolation and the masses of “foreigners” in the city. The observation 
of topic 3 being more present again due to the continuing dream cycle indicates this. Another 
interesting correlation with Lovecraft’s stay in Brooklyn comes from topic 13. This topic covers 
terms like ‘sea’ and ‘water’ that are somewhat lacking during his time away from Providence (1922 
to 1926, work 37 to 49). This comes from the fact that Lovecraft was inspired by Providence and 
the nearby sea, though Brooklyn was close to the sea too, it was the city that influenced Lovecraft’s 
themes at that time. For ‘The Dream Quest of Unknown Kadath’ (work 55), it’s probably the most 
important work of the dream cycle and covers the fictional character Randolph Carter, who was 
supposedly an alter ego of Lovecraft as mentioned earlier. The graph also suggests a lower share 
of the first topic during the timeframe for the works covered by that topic, work 56-61 from spring 
1927 to 1930. This correlates to the fact that he divorced in 1929 and there may have been first 
signs of a shift in themes and motifs caused by unknown problems in his and Greene’s relationship 
prior to the divorce. Topic 2 is present in some of the last works again as well as topic 1 having 
a higher share starting at work 62, ‘The Thing on the Doorstep’, which was written in the summer of 
1930. This and the higher share of topic 4 starting from work 65 (‘Dreams in the Witch House’, written 
1932) indicate an observable shift in topics following a break from writing because of the death of 
his aunt Mrs. Clark in 1932 and moving with his other aunt Mrs. Gamwell into small quarters in 
Providence. He was very close with aunt Mrs. Clark and her death could be represented in the shift 
starting at work 65 where topics 4 and 14 become highly present and topic 1 having a noticeably 
higher share from work 68. Another explanation for this mix of topics during the last years could 
be that Lovecraft had a hard time selling his longer stories and concentrated on ghostwriting and 
non-fiction, resulting in the fictional works he wrote being less frequent but longer and more 
focussed on different topics.