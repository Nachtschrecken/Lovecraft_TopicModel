\section{Related Work}

In this chapter, we will cover some related work that represents use cases of topic modeling 
similar to our research question. They analyze either Lovecraft's works or compare topics 
between Lovecraft's works and films and other authors.\\

The first related work is the paper 'Beyond the mountain of madness: a look at the shared 
themes of Edgar Allan Poe and H.P. Lovecraft' by Kristoffer Gustafson \cite{gusta}. This paper compares 
the main themes of both H.P. Lovecraft and Edgar Allan Poe, another well-known author, and 
poet. Poe heavily influenced Lovecraft's fiction and literary style, as he stated in his essay 
'Supernatural Horror in Literature'. The author examines the themes of insanity, death, and the 
gothic setting on both authors and projects similarities that suggest an influence of Poe on 
Lovecraft.\\

Another paper is 'The Lovecraft Look: An Examination of Lovecraftian Themes in Film' by 
Michael A. Church \cite{church}. The author analyzes the philosophical beliefs and life experiences that 
inspired Lovecraft's 'weird fiction' and his literary philosophy of 'Cosmic Indifferentism' 
and compares them with films influenced by Lovecraft. Lovecraft's philosophy has been heavily 
misconstrued, as evidenced by several films that purport to adapt his stories, but actually 
ignore or misinterpret Cosmic Indifferentism. However, some films successfully adhere to 
Lovecraft's focus on cosmic horror and humanity's insignificance, even if they are not direct 
adaptations of his work. The relevant task of uncovering topics gets applied in this project 
to compare Lovecraft's works and these films.\\

The third related work, 'Re-visioning Romantic-Era Gothicism: An Introduction to Key Works 
and Themes in the Study of H.P. Lovecraft' by Philip Smith \cite{smith}, is similar to our research 
question. The author examines the recurring themes of language, genre, literary influences, 
xenophobia, cosmic indifferentism, dreams, time, and the influence of Lovecraft. Contrary 
to our work, this paper focuses on the criticism of Lovecraft. As we said in our research 
agenda, Lovecraft was known for racist slurs and regarding our modern society, these have 
to be reflected. The paper does not cover all of Lovecraft's work and the key topics to be 
uncovered correlate to major critical responses.